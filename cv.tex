%git clone https://github.com/liantze/AltaCV.git CV
%http://mirrors.ibiblio.org/CTAN/fonts/fontawesome/doc/fontawesome.pdf
%pdflatex -> biber -> pdflatex -> pdflatex
\documentclass[10pt,a4paper,ragged2e,withhyper]{AltaCV/altacv}
\newif\ifEN\ENfalse
\newcommand{\huen}[2]{\ifEN#2\else#1\fi}
\ifEN
\else
%Magyar nyelvi támogatás (Babel 3.7 vagy későbbi kell!)
\def\magyarOptions{defaults=hu-min}
\usepackage[magyar]{babel}
\usepackage{t1enc}% ékezetes szavak automatikus elválasztásához
\fi

\geometry{left=1.25cm,right=1.25cm,top=1.5cm,bottom=1.5cm,columnsep=1.2cm}
\usepackage{paracol}
\ifxetexorluatex
% If using xelatex or lualatex:
\setmainfont{Roboto Slab}
\setsansfont{Lato}
\renewcommand{\familydefault}{\sfdefault}
\else
% If using pdflatex:
\usepackage[rm]{roboto}
\usepackage[defaultsans]{lato}
% \usepackage{sourcesanspro}
\renewcommand{\familydefault}{\sfdefault}
\fi
% Change the colours if you want to
\definecolor{szajam}{HTML}{693632}
\definecolor{SlateGrey}{HTML}{2E2E2E}
\definecolor{LightGrey}{HTML}{666666}
\definecolor{DarkPastelRed}{HTML}{517fb0}%Nagy cimek
\definecolor{PastelRed}{HTML}{24385d}%egyetemek diagram nyelvkorok idezet
\definecolor{GoldenEarth}{HTML}{d6b397}%vonalak
\colorlet{name}{szajam}
\colorlet{tagline}{PastelRed}
\colorlet{heading}{DarkPastelRed}
\colorlet{headingrule}{GoldenEarth}
\colorlet{subheading}{PastelRed}
\colorlet{accent}{PastelRed}
\colorlet{emphasis}{SlateGrey}
\colorlet{body}{LightGrey}
% Change some fonts, if necessary
\renewcommand{\namefont}{\Huge\rmfamily\bfseries}
\renewcommand{\personalinfofont}{\footnotesize}
\renewcommand{\cvsectionfont}{\LARGE\rmfamily\bfseries}
\renewcommand{\cvsubsectionfont}{\large\bfseries}
% Change the bullets for itemize and rating marker
% for \cvskill if you want to
\renewcommand{\itemmarker}{{\small\textbullet}}
\renewcommand{\ratingmarker}{\faCircle}
%% sample.bib contains your publications
\addbibresource{sample.bib}
\NewInfoField{locationandmailaddress}{\faMapMarker/\faEnvelope}
\NewInfoField{birthday}{\faHeartbeat}
\NewInfoField{languag}{\faLanguage}
\hypersetup{hidelinks}
\begin{document}
	%%%%%%%%%%%%%%%%%%%%%%%%%%%%%%%%%%%%%%%%%%%%%%%%%%%
	\name{\huen{Tasnádi Gábor}{Gábor Tasnádi}}
	\tagline{\faKeyboard \huen{Programtervező}{Software Engineer}}
	\photoL{3.5cm}{1}	
	\personalinfo{%
		\birthday{\huen{1997. 02. 04.}{4 Feb 1997}}
		\locationandmailaddress{\huen{1115 , Ballagi Mór utca 3. 2/14.}{3 Ballagi Mór Street, Budapest 1115}}
		\email{gabor.tasnadi@evosoft.com}
		\phone{+36 70 660 4792}
		\linkedin{\href{https://www.linkedin.com/in/tasnadigabor/}{tasnadigabor}}
		\github{\href{https://github.com/tasigabi97/cv}{tasigabi97/cv}}
		\languag{\huen{\href{https://github.com/tasigabi97/cv/blob/main/cv_en.pdf}{angolul}}{\href{https://github.com/tasigabi97/cv/blob/main/cv_hu.pdf}{hungarian}}}
		
	}
	\makecvheader
	%\AtBeginEnvironment{itemize}{\small}
	\medskip
	\medskip				
	\columnratio{0.6}
	\begin{paracol}{2}
		\cvsection{\huen{szakmai tapasztalat}{professional experience}}		
		\cvevent{\huen{Szoftverfejlesztő}{Software Developer}}{\huen{Szegedi Tudományegyetem}{University of Szeged}}{\huen{2019. 03. 01. -- 2019. 06. 30.}{1 Mar 2019 -- 30 Jun 2019}}{Szeged}
		\huen{Mély tanuló módszereket fejlesztettem arcképek szuper-rezolúciójára \textit{(biztonságtechnológiai rendszerhez)}.}{Developing deep learning methods for facial image super-resolution \textit{(for security systems)}.}\\
		\divider
		\cvevent{\huen{Tanszéki mérnök}{Engineer}}{\huen{Szegedi Tudományegyetem}{University of Szeged}}{\huen{2020. 09. 23. -- 2021. 01. 22.}{23 Sep 2020 -- 22 Jan 2021}}{Szeged \hfill \faGithub \href{https://github.com/tasigabi97/traffic}{tasigabi97/traffic}}
		\huen{Az egyetem informatikai karát népszerűsítettem egy digitális képfeldolgozó program készítésével \textit{(a pályázat címe: A Szegedi Tudományegyetem készségfejlesztő és kommunikációs programjainak megvalósítása a felsőoktatásba való bekerülés előmozdítására és az MTMI szakok népszerűsítésére)}.\\		
			A programom megszámolta, lokalizálta és azonosította a közúti közlekedés szereplőit – a róluk készített webkamerás felvételek alapján. Ezenkívül, az útburkolat jelölések automatikus kategorizálásával foglalkoztam.}{I promoted the Faculty of Informatics of the University with the creation of a digital image processing program \textit{(title of the competition: Promotion of the Faculty of Mathematics, Technical Sciences and Informatics of the University of Szeged)}.\\		
			My program is able to count, localize and identify the actors of road traffic based on their webcam footage. In addition, I dealt with the automatic categorization of pavement markers.}\\
		\medskip		
		\cvsection{\huen{Oktatás és képzés}{Education}}		
		\cvevent{\huen{Programtervező informatikus BSc}{Bachelor of Computer Science}}{\huen{Szegedi Tudományegyetem}{University of Szeged}}{}{}
		\textbf{\huen{Egyéb szabadon választott tárgyaim}{Additional optional subjects}:}
		\begin{itemize}[leftmargin=.25in]
			\item  Summer School on Image Processing (2019)
			\item \huen{Digitális topológia és matematikai morfológia}{Digital Topology and Mathematical Morphology}
			\item \huen{Bevezetés a mély tanulásba}{Introduction to Deep Learning}
		\end{itemize}
		\textbf{\huen{Szakdolgozatom témája}{Topic of my thesis}:}\\\noindent\hbox to .25in{}\huen{Mély konvolúciós hálók tanítása képfeldolgozási problémákra}{Creating of Deep Convolutional Neural Networks for Image\\\noindent\hbox to .25in{} Processing Problems}\\
		\textbf{\huen{TDK-dolgozatom témája}{Topic of my TDK dissertation}:}\\
		\noindent\hbox to .25in{}\huen{Hiperreflektív pontok szegmentálása OCT felvételeken}{Segmentation of Hyperreflective Foci on OCT Images}\\
		\divider
		\cvevent{Webdesigner}{\huen{József Attila Gimnázium}{József Attila High School}}{\huen{2014. 04. 17. -- 2015. 02. 19.}{17 Apr 2014 -- 19 Feb 2015}}{Monor \hspace{2em} \href{https://github.com/tasigabi97/cv/blob/main/webdesigner_certificate.jpg}{\faFileImage}}
		\huen{124 órás akkreditált képzés (HTML\slash CSS\slash JavaScript) 95\%-os eredménnyel.}{124-hour accredited course (HTML\slash CSS\slash JavaScript), 95\% result}\\	
	\switchcolumn
		\cvsection{\huen{Életfilozófiám}{My Life Philosophy}}
		
		\begin{quote}
			``\huen{Bármikor beüthet a siker, ha kellő ideig kitartunk}{Success is walking from failure to failure with no loss of enthusiasm.}''
		\end{quote}
		
		\cvsection{\huen{{\Large Amikre büszke vagyok}}{Most Proud of}}
		
		\cvachievement{\faTrophy}{\huen{TDK 2020. tavasz}{TDK Spring 2020}}{\huen{II. hely és Morgan Stanley különdíj}{2nd place and Morgan Stanley special award} \href{https://www.inf.u-szeged.hu/kutatas/konferenciak/tdk-2020-tavasz}{\faLink}}
		\divider\\
		\cvachievement{\faGraduationCap}{\huen{Jeles diploma}{Excellent degree}}{\huen{196 kredit és 4,74-es átlag}{196 credit and 4.74 average} \href{https://github.com/tasigabi97/cv/blob/main/degree_certificate_en.jpg}{\faFileImage}}\\
		\medskip
		\cvsection{\huen{{\Large hobbi és érdeklődés}}{{\Large hobbies and interests}}}
		\wheelchart{1cm}{0.2cm}{%
			4/8em/accent!70/{\huen{Számítógépes\\látás}{Computer\\vision}},
			3/8em/accent!40/{Raspberry Pi 4},
			4/8em/accent!60/{Tensorflow},
			1/8em/accent!30/{{\large \LaTeX}},
			1/10em/accent/{Docker},
			2/6em/accent!10/{Linux}
		}
		\cvsection{\huen{kompetenciák}{competences}}
		\cvtag{Latex}
		\cvtag{MS Office}
		\cvtag{Linux \& Windows}\\
		\divider\smallskip\\
		\huen{
			\cvtag{Jó kommunikációs készség}			\cvtag{Tanulékonyság}
			\cvtag{Pontosság}
			\cvtag{Csapatjátékos}
			\cvtag{Rugalmasság}
			\cvtag{Terhelhetőség}
			\cvtag{Motivált}\\
			\cvtag{Önálló munkára való képesség}}{%
			\cvtag{Good communication skills}
			\cvtag{Teachable}
			\cvtag{Punctual}
			\cvtag{Team player}
			\cvtag{Flexible}
			\cvtag{Motivated}
			\cvtag{Autonomous}
		}
		\\
		\divider\smallskip\\
		\cvtag{\huen{Általános / orvosi képfeldolgozás ismerete}{General / medical image processing}}\\
		\cvtag{\huen{Gépi tanulás ismerete}{Machine learning}}\\
		\cvsection{\huen{Nyelvek}{Languages}}
		\cvskill{\huen{Angol}{English} (B2\slash telc) \href{https://github.com/tasigabi97/cv/blob/main/language_certificate.jpg}{\faFileImage}}{3}
		\divider
		\cvskill{Python}{5}\smallskip
		\cvskill{C\slash C++}{2}
		\cvskill{C\#\slash Java}{2}
		\cvskill{(PL/)SQL}{2}
		\divider
		\medskip		
	\end{paracol}
\end{document}

 
